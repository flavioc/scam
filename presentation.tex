\documentclass[11pt]{beamer}
%\documentclass[handout,t]{beamer}

\batchmode
\usepackage[svgnames]{}
\usepackage{pgfpages}
\usepackage{listings}
\usepackage{relsize}
\usepackage{verbatim}
% \pgfpagesuselayout{4 on 1}[letterpaper,landscape,border shrink=5mm]

\usepackage{amsmath,amssymb,enumerate,epsfig,bbm,calc,color,ifthen,capt-of}
\usepackage[english]{babel}
\usepackage[utf8]{inputenc}

\usetheme{Berlin}
%\usecolortheme{bear}

\title{AspectJ: An Approach to AOP}
\author{Carlos Pereira \and Flávio Cruz}

\date{January the 17th, 2011}

\institute[Source Code Analysis and Manipulation]{Source Code Analysis and Manipulation}

%Este template é apenas um exemplo e as secções sao apenas ilustrativas...Se preferires o "tradicional", por mim tudo bem :)

% -----------------------------------------------------------------------------
\begin{document}
% -----------------------------------------------------------------------------

\frame{\titlepage}

\section[Summary]{}
\subsection*{}
\begin{frame}{Summary}
  \tableofcontents
\end{frame}

%----------------------------AOP----------------------------------------------------
\section{Aspect Oriented Programming}
\subsection*{Aspect Oriented Programming}

\begin{frame}{Aspect Oriented Programming}
\begin{itemize}
\item AOP tries to tackle the problem of \emph{crosscutting concerns}  that is usually found in paradigms such as Object Oriented Programming (OOP) and Imperative Programming
\item These concerns defy well-known abstractions like classes, methods or functions
\item They crosscut the natural modularity present in the implementation of complex systems because they tend to be dispersed in several modular units
\end{itemize}
\end{frame}

\begin{frame}{Crosscutting concerns}
\begin{itemize}
   \item Crosscutting concerns result in code duplication and scattering making the understanding of a code-base much harder
   \item This is a violation of a golden rule in software engineering: separation of concerns
   \item Moreover, the main program logic gets lost in the middle of code that is not very important to the understanding of a system
   \item Examples of crosscutting concerns are:
      \begin{itemize}
         \item Logging
         \item Authentication
         \item Resource management: memory, files ...
         \item Event dispatching and handling
      \end{itemize}
\end{itemize}
\end{frame}

\begin{frame}{Objectives of AOP}
\begin{itemize}
\item Provide new means of abstraction that can explicitly capture crosscutting concerns
\item ... and separate them from the \emph{core concerns}
\item This abstraction is called the \emph{aspect} and enables the programmer to
   \begin{itemize}
      \item \textbf{Develop} crosscutting concerns
      \item \textbf{Maintain} crosscutting concerns easily in one place
      \item \textbf{Reuse} crosscutting concerns
   \end{itemize}
\item With aspects, the programmer can focus on the core concerns and deal with crosscutting concerns in a modular way
\end{itemize}
\end{frame}

\begin{frame}{Join Point Model}
\begin{itemize}
   \item In AOP based languages, the aspect abstraction is defined by a \emph{join point model}
   \item This model is composed of several concepts:
      \begin{itemize}
         \item \emph{Join points}: well-defined points in the execution of the program
         \item \emph{Pointcuts}: groups of joint points
         \item \emph{Advices}: method-like constructs used to implement the crosscutting concerns
      \end{itemize}
\end{itemize}
\end{frame}

\begin{frame}{Join Point Model}
\begin{itemize}
   \item An aspect is thus an unit of modular crosscutting implementation that is composed of pointcuts and
the corresponding advices
   \item We can thus use pointcuts associated with advices to allow the execution of secondary code at specific parts of the program
   \item The process of "mixing" standard code with aspects is called \emph{aspect weaving}
\end{itemize}
\end{frame}

%----------------------------AspectJ-------------------------------------------------------------------------------
\section{AspectJ}
\subsection*{AspectJ}

\begin{frame}{AspectJ}
  \begin{itemize}
    \item AspectJ is the Java implementation of AOP.
    \item Extends the Java grammar by adding constructs that enable the modular implementation of crosscutting concerns.
    \item Upholds two main properties:
    \begin{itemize}
    	\item \emph{Upward compatibility} - all legal Java programs must be leval AspectJ programs.
      	\item \emph{Platform compatibility} - all legal AspectJ programs must run on standard Java virtual machines.
     \end{itemize}
  \end{itemize}
\end{frame}

\begin{frame}{Ajc}
  \begin{itemize}
    \item Ajc is the AspectJ compiler/weaver
    \item A weaver needs to ``link'' together classes and aspects so that advice gets executed,
inter-type declarations affect the static structure, and weave-time declarations produce
warnings and errors.
    \item AspectJ offers three weaving models:
    \begin{itemize}
      \item Source weaving
      \item Binary weaving
      \item Load-time weaving
    \end{itemize}
  \end{itemize}
\end{frame}

\begin{frame}{Ajc}
  \begin{itemize}
    \item \emph{Source weaving} - the input to the weaver consists of classes and
aspects in source-code form.
    \item \emph{Binary weaving} - input to the weaver - classes and aspects - is in byte-code form. The
input byte code is compiled separately using the Java compiler or the AspectJ compiler.
    \item \emph{Load-time weaving} - takes input in the form of binary classes and aspects, as well as
aspects and configuration defined in XML format.
    \item Regardless of the weaving model used, the resulting execution of the system is identical.
  \end{itemize}
\end{frame}

%---------------------------Join Point Model in AspectJ----------------------------------------------------
\section{Join Point Model}
\subsection*{Join Point Model}

\begin{frame}{Join Point Model}

%Add descriptions of each on comment for reference...
AspectJ can possibly detect and operate on the following eleven kinds of join points:

\begin{columns}
  \column{0.55\textwidth}
  \begin{itemize}
    \item method call 
    \item method execution 
    \item constructor call 
    \item initializer execution 
    \item constructor execution 
    \item static initializer execution 
  \end{itemize}

  \column{0.45\textwidth}
  \begin{itemize}
    \item object preinitialization 
    \item object initialization 
    \item field reference 
    \item field assignment 
    \item exception handler execution
  \end{itemize}
\end{columns}
\end{frame}

\begin{frame}{Pointcut Designator}
  \begin{itemize}
    \item A pointcut designator can be composed of several join points in addition to values inherent to the execution context of those join points.
    \item AspectJ allows the implementation of two types of pointcuts: \emph{primitive pointcuts} and \emph{user-defined pointcuts}. 
    \item Primitive pointcuts are used to identify join points supported by AspectJ. 
    \item User-defined pointcuts allow the composition of primitive pointcuts. 
  \end{itemize}
\end{frame}

\begin{frame}{Primitive Pointcuts}
  \begin{itemize}
    \item \texttt{call(MethodPattern $||$ ConstructorPattern)}: 
    Picks out each method call join point whose signature matches MethodPattern or ConstructorPattern. 
    \item \texttt{execution(MethodPattern $||$ ConstructorPattern)}:
    Picks out each method execution join point whose signature matches MethodPattern or ConstructorPattern. 
    \item \texttt{get(FieldPattern)}: 
    Picks out each field reference join point whose signature matches FieldPattern. 
    \item \texttt{set(FieldPattern)}:
    Picks out each field set join point whose signature matches FieldPattern. 
  \end{itemize}
\end{frame}

\begin{frame}{Primitive Pointcuts}
  \begin{itemize}
    \item \texttt{initialization(ConstructorPattern)}:
    Picks out each object initialization join point whose signature matches ConstructorPattern. 
    \item \texttt{preinitialization(ConstructorPattern)}:
    Picks out each object pre-initialization join point whose signature matches ConstructorPattern. 
    \item \texttt{handler(TypePattern)}:
    Picks out each exception handler join point whose signature matches TypePattern. 
  \end{itemize}
\end{frame}

\begin{frame}{Primitive Pointcuts}
  \begin{itemize}
    \item \texttt{within(TypePattern)}:
    Picks out each join point where the executing code is defined in a type matched by TypePattern. 
    \item \texttt{withincode(MethodPattern $||$ ConstructorPattern)}:
    Picks out each join point where the executing code is defined in a method whose signature matches MethodPattern or a Constructor Pattern. 
    \item \texttt{cflow(Pointcut)}:
    Picks out each join point in the control flow of any join point P picked out by Pointcut, including P itself. 
  \end{itemize}
\end{frame}

\begin{frame}{Primitive Pointcuts}
  \begin{itemize}
    \item \texttt{this(Type or Id)}:
    Picks out each join point where the currently executing object (the object bound to this) is an instance of Type, or of the type of the identifier Id. 
    \item \texttt{target(Type or Id)}:
    Picks out each join point where the target object (the object on which a call or field operation is applied to) is an instance of Type, or of the type of the identifier Id.
    \item \texttt{args(Type or Id, ...)}:
    Picks out each join point where the arguments are instances of the appropriate type (or type of the identifier if using that form). 
  \end{itemize}
\end{frame}


\begin{frame}{User-defined Pointcuts}
  \begin{itemize}
    \item User-defined pointcuts define new pointcut designators using combinations of primitive or user-defined pointcuts.
    \item Special operator can be applied to the pointcut descriptors, such as \texttt{||} (logical or), \texttt{!} (logical not) and \texttt{\&\&} (logical and)
  \end{itemize}
  \begin{block}{An example:}
    \texttt{pointcut pointcutOne() : execute (* methodOne(..)) \&\& execute(* methodTwo(..));}
  \end{block}
\end{frame}

\begin{frame}{Advice}
  \begin{itemize}
    \item An advice defines what code should run at join points that are picked out by pointcuts.
    \item An advice can be triggered by named or nameless pointcuts and can have formal parameters that are either provided by the pointcuts or exposed by the advice itself.
    \item The three basic kinds of advices are \emph{before}, \emph{after}, and \emph{around}.
  \end{itemize}
\end{frame}

\begin{frame}{Advice}
  \begin{itemize}
    \item \emph{before} - the advice’s body is executed before the join point picked by its pointcut is reached. 
    \item \emph{after} - the goal of the after advice is to execute its body after the corresponding join pointing. It is possible to indicate if whether the advice is meant to run if the join point has returned (\emph{after returning}) or if it has thrown an exception (\emph{after throwing}).
    \item \emph{around} - the around advice enables the developer to replace the join point with an arbitrary code. The statement \emph{proceed()} can be used to indicate that the replaced join point code can be executed.
  \end{itemize}
\end{frame}

\begin{frame}{Aspect}
  \begin{itemize}
    \item Aspects represent the basis for AOP as the principal items for crosscutting implementations on which all the pointcuts and their corresponding advices are declared.
    \item Aspects can be implemented similarly to Java classes, possessing properties like class extension, interface implementation or inner declarations.
    \item The AspectJ compiler translates aspects into Java classes.
  \end{itemize}
\end{frame}

\begin{frame}{Aspect Instantiation}
   \begin{itemize}
      \item AspectJ supports both singleton aspects and per object aspects.
      \item Singleton aspects are instantiated only once and used in the whole program.
      \begin{itemize}
         \item Data members and methods of the aspect are accessed by using the method \texttt{Aspect.aspectOf()}.
      \end{itemize}
      \item For per object aspects, the aspect is instantiated for each object and each object can access the aspects data members and methods.
      \begin{itemize}
         \item The method \texttt{Aspect.aspectOf(object)} is used to access data and methods.
      \end{itemize}
   \end{itemize}
\end{frame}

\begin{frame}{Reflective Access}
   \begin{itemize}
      \item It is possible to access some information about the current join point.
      \item AspectJ generates objects that encapsulate some of the context of the advice's current or enclosing join point.
      \begin{itemize}
         \item \texttt{thisJoinPoint}: bound to a complete join point object.
         \item \texttt{thisJoinPointStaticPart}: static part of the previous object.
         \item \texttt{thisEnclosingJoinPointStaticPart}: static part of the enclosing join point.
      \end{itemize}
   \end{itemize}
\end{frame}

\begin{frame}{Extension of classes}
   \begin{itemize}
      \item Aspects also allow the extension and modification of the class hierarchy.
      \begin{itemize}
         \item By changing the superclass.
         \item Adding an interface.
      \end{itemize}
      \item Syntax:
      \begin{itemize}
         \item \texttt{declare parents: TypePattern extends Type;}
         \item \texttt{declare parents: TypePattern implements TypeList;}
      \end{itemize}
      \item It also possible to define new class methods right after a \texttt{declare parents} line.
   \end{itemize}
\end{frame}

%------------------------------Video (?)-------------------------------------------------------------
\section{Demonstration}
\subsection*{Demonstration}

\begin{frame}{Video demonstration}
Topics on the video:
\begin{itemize}
\item Installation
\item Compilation
\item Program examples
\end{itemize}
\end{frame}

\begin{frame}{Uses of AOP}
\begin{itemize}
   \item As we know from previous information, AOP is used to tackle the problem of crosscutting concerns
   \item Obviously, AOP is good for things like execution tracing, logging, authentication, etc
   \item With execution tracing it is possible to make debugging easier since debug code is not mixed with everything else
\end{itemize}
\end{frame}

\begin{frame}{Uses of AOP}
\begin{itemize}
   \item Another good use of AOP is to implement the so-called \emph{design by contract} or \emph{defensive programming}
   \begin{itemize}
      \item Things like invariants, preconditions or post conditions are easily implementable using aspects
      \item They make sure that the function is receiving the right inputs and generating the right outputs - useful for testing
   \end{itemize}
   \item AOP is also notably good to make clear and concise implementations of some design patterns
      \begin{itemize}
         \item Adapter
         \item Observer
         \item ...
         \item A researcher noted that 17 out of the 23 GoF patterns exhibited some degree of crosscutting %Design Patterns: Elements of Reusable Object-Oriented Software
      \end{itemize}
\end{itemize}
\end{frame}

\subsection*{Observer Pattern}
\begin{frame}{Observer Pattern}
\begin{itemize}
   \item This is one of the GoF patterns
   \item An object, called the subject, keeps a list of observers
   \item When something in the subject changes, the subject notifies, automatically, all its observers
   \item This pattern is used to implement distributed event handling systems
\end{itemize}
\end{frame}

\begin{frame}{Observer Pattern}
\begin{itemize}
   \item Let's implement the Observer Pattern in a drawing system:
      \begin{itemize}
         \item Observer: \texttt{Display}
         \item Subject: \texttt{Shape}
      \end{itemize}
\end{itemize}
\end{frame}

\begin{frame}[fragile]
   \frametitle{Observer - Coord}
   {\small
   \begin{lstlisting}[language=java]
   public class Coord
   {
      public int x, y;

      public Coord(int _x, int _y) {
         x = _x;
         y = _y;
      }

      public String toString() {
         return "(" + x + ", " + y + ")";
      }
   }
   \end{lstlisting}}
\end{frame}

\begin{frame}[fragile]
   \frametitle{Observer - Shape}
   {\footnotesize
      \begin{lstlisting}[language=java]
public abstract class Shape
{
   public abstract void draw(Coord coord);
}
      \end{lstlisting}
   }
\end{frame}

\begin{frame}[fragile]
   \frametitle{Observer - Square}
   {\footnotesize
   \begin{lstlisting}[language=java]
public class Square extends Shape
{
  private int side;

  public Square(int _side) {
    side = _side;
  }

  public void draw(Coord coord) {
    System.out.println("Drawing square(side="
          + side + ") in " + coord);
  }
}
   \end{lstlisting}
   }
\end{frame}

\begin{frame}[fragile]
   \frametitle{Observer - Circle}
   {\footnotesize
   \begin{lstlisting}[language=java]
public class Circle extends Shape
{
  private int radius;

  public Circle(int _radius) {
    radius = _radius;
  }

  public void draw(Coord coord) {
    System.out.println("Drawing a circle(radius=" +
               radius + ") in " + coord);
  }
}
   \end{lstlisting}
   }
\end{frame}

\begin{frame}[fragile]
   \frametitle{Observer - Display}
   {\footnotesize
   \begin{lstlisting}[language=java]
public class Display
{
  public Display() {
  }

  public void update() {
    System.out.println("Updating display...");
  }
}
   \end{lstlisting}
   }
\end{frame}

\begin{frame}[fragile]
   \frametitle{Observer - ObserverPattern}
   {\tiny
   \begin{lstlisting}[language=java]
public abstract aspect ObserverPattern perthis(subjectConstructed(Subject))
{
  public interface Observer {}
  public interface Subject {}

  private Vector observers = new Vector();
  public void addObserver(Observer o) {
    observers.add(o);
  }

  protected pointcut subjectConstructed(Subject s) :
    this(s) && execution(Subject+.new(..));

  abstract protected pointcut subjectChanged(Subject s);
  after(Subject s) : subjectChanged(s) {
    Iterator itr = observers.iterator();
    while(itr.hasNext()) {
      updateObserver((Observer)itr.next(), s);
    }
  }

  public abstract void updateObserver(Observer o, Subject s);
}
   \end{lstlisting}
   }
\end{frame}

\begin{frame}[fragile]
   \frametitle{Observer - ShapeObserverPattern}
   {\footnotesize
   \begin{lstlisting}[language=java]
public aspect ShapeObserverPattern extends ObserverPattern
{
  declare parents: Shape implements Subject;
  declare parents: Display implements Observer;

  protected pointcut subjectChanged(Subject s) :
    this(s) && execution(void Shape.draw(Coord));

  public void updateObserver(Observer o, Subject s) {
    ((Display)o).update();
  }
}
   \end{lstlisting}
   }
\end{frame}

\begin{frame}[fragile]
   \frametitle{Observer - Now to the main program}
   {\scriptsize
\begin{lstlisting}[language=java]
public class Main
{
  public static void main(String[] args) {
   Display display = new Display();
   Square sq = new Square(2);
   Circle circle = new Circle(5);

   ShapeObserverPattern.aspectOf(sq).addObserver(display);
   ShapeObserverPattern.aspectOf(circle).addObserver(display);

   sq.draw(new Coord(0, 0));
   circle.draw(new Coord(10, 5));
  }
}
\end{lstlisting}
   }
\end{frame}

\begin{frame}[fragile]
   \frametitle{Observer - Execution}
   {\scriptsize
\begin{lstlisting}
$ ajc Coord.java Display.java \
      Shape.java Square.java Circle.java \
      ObserverPattern.java ShapeObserverPattern.java \
      Main.java
$ java Main
Drawing square(side=2) in (0, 0)
Updating display...
Drawing a circle(radius=5) in (10, 5)
Updating display...
$
\end{lstlisting}
   }
\end{frame}

\subsection*{Tracing}
\begin{frame}[fragile]
   \frametitle{Tracing - Trace}
   {\scriptsize
   \begin{lstlisting}[language=java]
import org.aspectj.lang.Signature;

abstract aspect Trace {
  private static int callDepth = 0;

  private static void printMethod(Signature sign,Object[] args) {
    System.out.print(sign.getName() + "(");
    for(int i = 0; i < args.length; ++i) {
      System.out.print(args[i].toString());
      if(i != args.length - 1)
        System.out.print(", ");
    }

    System.out.print(")");
  }
   \end{lstlisting}}
\end{frame}

\begin{frame}[fragile]
   \frametitle{Tracing - Trace}
   {\scriptsize
   \begin{lstlisting}[language=java]
private static void printEntering(Signature sign, Object[] args){
  printIndent();
  System.out.print("-> ");
  printMethod(sign, args);
  System.out.println("");
}
private static void printExiting(Signature sign, Object[] args,
      Object ret) {
  printIndent();
  System.out.print("<- ");
  printMethod(sign, args);
  System.out.println(" returning " + ret.toString());
}
private static void printIndent() {
  for(int i = 0; i < callDepth; ++i)
    System.out.print(" ");
}
   \end{lstlisting}}
\end{frame}

\begin{frame}[fragile]
   \frametitle{Tracing - Trace}
   {\scriptsize
   \begin{lstlisting}[language=java]
abstract pointcut myClass();

pointcut constructorPC(): myClass() && execution(new(..));
pointcut methodPC(): myClass() && execution(* *(..));

before(): constructorPC() || methodPC() {
  printEntering(thisJoinPoint.getSignature(),
      thisJoinPoint.getArgs());
  callDepth++;
}

after() returning (Object ret): constructorPC() || methodPC() {
  callDepth--;
  printExiting(thisJoinPoint.getSignature(),
      thisJoinPoint.getArgs(), ret);
}
   \end{lstlisting}}
\end{frame}

\begin{frame}[fragile]
   \frametitle{Tracing - MyClass}
   {\tiny
   \begin{lstlisting}[language=java]
public class MyClass {
  private int ab;

  private int doFactorial(int n) {
    if(n <= 1)
      return 1;
    return n * doFactorial(n-1);
  }
  public int factorial() {
    return doFactorial(ab);
  }

  private int doFib(int n) {
    if(n <= 1)
      return n;
    return doFib(n-1) + doFib(n-2);
  }
  public int fib() {
    return doFib(ab);
  }

  public MyClass(int a, int b) {
    ab = a * b;
  }
}
   \end{lstlisting}}
\end{frame}

\begin{frame}[fragile]
   \frametitle{Tracing - ClassTrace}
   {\small
   \begin{lstlisting}[language=java]
public aspect ClassTrace extends Trace {
  pointcut myClass(): within(MyClass);
}
   \end{lstlisting}}
\end{frame}

\begin{frame}[fragile]
   \frametitle{Tracing - Main}
   {\footnotesize
   \begin{lstlisting}[language=java]
public class Main {
  public static void main(String[] args) {
    MyClass obj1 = new MyClass(2, 2);
    MyClass obj2 = new MyClass(3, 1);

    System.out.println(obj1.factorial());
    System.out.println(obj2.fib());
  }
}
   \end{lstlisting}}
\end{frame}

\begin{frame}[fragile]
   \frametitle{Tracing - Execution}
   {\scriptsize
   \begin{lstlisting}
-> <init>(2, 2)
 -> <init>(3, 1)
  -> factorial()
   -> doFactorial(4)
    -> doFactorial(3)
     -> doFactorial(2)
      -> doFactorial(1)
      <- doFactorial(1) returning 1
     <- doFactorial(2) returning 2
    <- doFactorial(3) returning 6
   <- doFactorial(4) returning 24
  <- factorial() returning 24
24
   \end{lstlisting}}
\end{frame}

\begin{frame}[fragile]
   \frametitle{Tracing - Execution}
   {\scriptsize
   \begin{lstlisting}
  -> fib()
   -> doFib(3)
    -> doFib(2)
     -> doFib(1)
     <- doFib(1) returning 1
     -> doFib(0)
     <- doFib(0) returning 0
    <- doFib(2) returning 1
    -> doFib(1)
    <- doFib(1) returning 1
   <- doFib(3) returning 2
  <- fib() returning 2
2
   \end{lstlisting}}
\end{frame}

\subsection*{Programming by Contract}

\begin{frame}[fragile]
   \frametitle{Programming by Contract}
   \begin{itemize}
      \item Let's see another AspectJ example, now for defensive programming / programming by contract.
   \end{itemize}
   {\scriptsize
   \begin{lstlisting}[language=java]
public class Customer {
}

public class CompanySystem {
   Customer getCustomer(String name) {
      return new Customer();
   }
}
   \end{lstlisting}}
   
\end{frame}

\begin{frame}[fragile]
   \frametitle{Programming by Contract - ContractBrokeException}
{\scriptsize
\begin{lstlisting}[language=java]
public class ContractBrokeException extends RuntimeException {
  public ContractBrokeException(String arg0) {
    super(arg0);
  }

  public ContractBrokeException(Throwable arg0) {
    super(arg0);
  }

  public ContractBrokeException(String arg0, Throwable arg1) {
    super(arg0, arg1);
  }
}
\end{lstlisting}}
\end{frame}

\begin{frame}[fragile]
   \frametitle{Programming by Contract - AbstractContract}
{\scriptsize
\begin{lstlisting}[language=java]
public abstract aspect AbstractContract {
  public abstract pointcut targetPointcut();

  public abstract ContractManager getContractManager();

  Object around(): targetPointcut() {
    ContractManager manager = getContractManager();
    System.out.println("Checking contract with " +
        manager.getClass().getName());

    if(manager != null) {
      System.out.println("Initial invariants check");
      manager.checkInvariants(thisJoinPoint.getTarget());

      System.out.println("Pre-conditions check");
      manager.checkPreConditions(thisJoinPoint.getTarget(),
          thisJoinPoint.getArgs());
    }
\end{lstlisting}}
\end{frame}

\begin{frame}[fragile]
   \frametitle{Programming by Contract - AbstractContract}
{\scriptsize
\begin{lstlisting}[language=java]
    Object obj = proceed();

    if(manager != null) {
      System.out.println("Post-conditions check");
      manager.checkPostConditions(thisJoinPoint.getTarget(),
          obj, thisJoinPoint.getArgs());

      System.out.println("Final invariants check");
      manager.checkInvariants(thisJoinPoint.getTarget());
    }
    return obj;
  }
}
\end{lstlisting}}
\end{frame}

\begin{frame}[fragile]
   \frametitle{Programming by Contract - ContractManager}
{\tiny
\begin{lstlisting}[language=java]
public interface ContractManager
{
  public void checkPreConditions(Object thisObject, Object[] args)
    throws ContractBrokeException;

  public void checkPostConditions(Object thisObject, Object returnValue,
      Object[] args)
    throws ContractBrokeException;

  public void checkInvariants(Object thisObject) throws ContractBrokeException;
}
\end{lstlisting}}
\end{frame}

\begin{frame}[fragile]
   \frametitle{Programming by Contract - CompanySystemContractManager}
{\tiny
\begin{lstlisting}[language=java]
public class CompanySystemContractManager implements ContractManager {
  public void checkPreConditions(Object thisObject, Object[] args)
    throws ContractBrokeException {
    if(args[0] == null)
      throw new ContractBrokeException("Pre Error: argument must be non-null");
  }

  public void checkPostConditions(Object thisObject, Object value,
      Object[] args) throws ContractBrokeException {
    if(value == null)
      throw new ContractBrokeException("Post Error: return value must be non-null");
  }

  public void checkInvariants(Object thisObject) throws ContractBrokeException {
  }
}
\end{lstlisting}}
\end{frame}

\begin{frame}[fragile]
   \frametitle{Programming by Contract - ContractCompanySystem}
{\scriptsize
\begin{lstlisting}[language=java]
public aspect ContractCompanySystem extends AbstractContract {
  public pointcut targetPointcut():
      call(Customer CompanySystem.getCustomer(String));

  public ContractManager getContractManager() {
    return new CompanySystemContractManager();
  }
}
\end{lstlisting}}
\end{frame}

\begin{frame}[fragile]
   \frametitle{Programming by Contract - Main}
{\scriptsize
\begin{lstlisting}[language=java]
public class Main {
  public static void main(String[] args) {
    CompanySystem csystem = new CompanySystem();
    Customer customer = csystem.getCustomer("joe");
  }
}
\end{lstlisting}}
\end{frame}

\begin{frame}[fragile]
   \frametitle{Programming by Contract - Compilation}
{\scriptsize
\begin{lstlisting}
$ ajc Customer.java CompanySystem.java \
      AbstractContract.java   \
      ContractManager.java \
      ContractBrokeException.java \
      CompanySystemContractManager.java \
      ContractCompanySystem.java \
      Main.java
$ 
\end{lstlisting}}
\end{frame}

\begin{frame}[fragile]
   \frametitle{Programming by Contract - Success}
{\scriptsize
\begin{lstlisting}
$ java Main
Checking contract with CompanySystemContractManager
Initial invariants check
Pre-conditions check
Post-conditions check
Final invariants check
$ 
\end{lstlisting}}
\end{frame}

\begin{frame}[fragile]
   \frametitle{Programming by Contract - Failure}
{\scriptsize
\begin{lstlisting}
$ java Main
Checking contract with CompanySystemContractManager
Initial invariants check
Pre-conditions check
Post-conditions check
Exception in thread "main" ContractBrokeException:
      Post Error: return value must be non-null
	at CompanySystemContractManager.checkPostConditions(
	   CompanySystemContractManager.java:12)
	at Main.getCustomer_aroundBody1$advice(Main.java:25)
	at Main.main(Main.java:5)
$ 
\end{lstlisting}}
\end{frame}

%------------------------------Technical Details-----------------------------------------------------

\section{Technical Details}
\subsection*{Technical Details}

\begin{frame}{Technical Details}
\begin{itemize}
   \item Being mostly a specification, AspectJ does not enforce any particular implementation strategy
   \item However, the tool itself tries to do most of the aspect weaving during compile time
   \item This allows the compiler to catch errors earlier and increase runtime efficiency
   \item Code unaffected by aspects is compiled as usual, while code where advices apply is transformed
   to include static points that correspond to the dynamic join points
\end{itemize}
\end{frame}

\begin{frame}{Technical Details}
\begin{itemize}
   \item Before and after advices are compiled as standard methods and are called in the static points
   \item Around advices are compiled into multiple methods, one for each static point in the code.
   \begin{itemize}
      \item This increases code size but increases runtime efficiency since it is faster to access data through the call stack
      \item The \texttt{proceed} statement is much easier to implement
   \end{itemize}
   \item Dynamic checks are used for dynamic join points
   \item Resulting byte-code can be executed in any JVM.
\end{itemize}
\end{frame}


% -----------------------------Conclusions-----------------------------------------------------------------------
\section{Conclusions}
\subsection*{}
\begin{frame}{Conclusions}
\begin{itemize}
   \item AOP attempts to solve shortcomings of the current widely used paradigms.
   \begin{itemize}
      \item Crosscutting concerns.
      \item Separation of concerns.
   \end{itemize}
   \item The AspectJ join point model allows the flexible definition of pointcuts and corresponding advices. 
   \item We think that AspectJ is a highly complex tool with lots of options...
   \begin{itemize}
      \item But targeting Java is nice to bring AOP to the average programmer.
      \item Outputting standard JVM byte-code is a nice plus.
   \end{itemize}
\end{itemize}
\end{frame}

%-------------------------------Last Slide-----------------------------------------------------------
\section*{}
\frame{\titlepage}

% -----------------------------------------------------------------------------
\end{document}
